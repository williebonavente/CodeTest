\documentclass[11pt]{article}
\usepackage[margin=3cm]{geometry}
\usepackage{algorithm2e}

\begin{document}

Here we demonstrate how algorithms or pseudocode can be typeset using the \verb|algorithm| environment provided by the \verb|algorithm2e| package.

    {\itshape You should not load the \verb|algorithm|, \verb|algpseudocode|, \verb|algcompatible|, \verb|algorithmic| packages if you have already loaded \verb|algorithm2e|.}

Note that the command and argument syntax provided by \verb|algorithm2e| are very different from those provided by \verb|algpseudocode|. It is important to know clearly which package that you are using, and then accordingly write the relevant commands with the correct syntax.

\begin{algorithm}
    $i\gets 10$\;
    \eIf{$i\geq 5$}
    {
        $i\gets i-1$\;
    }{
        \If{$i\leq 3$}
        {
            $i\gets i+2$\;
        }
    }
\end{algorithm}

Every line in your source code \textbf{must} end with \verb|\;| otherwise your algorithm will continue on the same line of text in the output. Only lines with a macro beginning a block should not end with \verb|\;|.

The above algorithm example is uncaptioned. If you need a caption for your algorithm, use \verb|\caption{...}| inside the \verb|algorithm| environment.
You can then use \verb|\label{...}| after the \verb|\caption| so that the algorithm number can be cross-referenced, e.g.~Algorithm~\ref{alg:two} and \ref{alg:three}.

By default, the \verb|plain| algorithm style is used. But if you prefer lines around the algorithm and caption, you can add the \verb|ruled| package option when loading \verb|algorithm2e|, or write \verb|\RestyleAlgo{ruled}| in your document.

\RestyleAlgo{ruled}

%% This is needed if you want to add comments in
%% your algorithm with \Comment
\SetKwComment{Comment}{/* }{ */}

\begin{algorithm}[hbt!]
    \caption{An algorithm with caption}\label{alg:two}
    \KwData{$n \geq 0$}
    \KwResult{$y = x^n$}
    $y \gets 1$\;
    $X \gets x$\;
    $N \gets n$\;
    \While{$N \neq 0$}{
        \eIf{$N$ is even}{
            $X \gets X \times X$\;
            $N \gets \frac{N}{2} $ \Comment*[r]{This is a comment}
        }{\If{$N$ is odd}{
                $y \gets y \times X$\;
                $N \gets N - 1$\;
            }
        }
    }
\end{algorithm}


% Bubble Sort Algorithm
\begin{algorithm}[hbt!]
    \caption{Bubble Sort Algorithm}\label{alg:bubble_sort}
    \KwData{An array $A$ of $n$ elements}
    \KwResult{The array $A$ sorted in non-decreasing order}
    \For{$i \gets 0$ \KwTo $n-1$}{
        \For{$j \gets 0$ \KwTo $n-i-1$}{
            \If{$A[j] > A[j+1]$}{
                swap $A[j]$ and $A[j+1]$\;
            }
        }
    }
\end{algorithm}

% Binary Search Algorithm

\begin{algorithm}[hbt!]
    \caption{Binary Search Algorithm}\label{alg:binary_search}
    \KwData{An array $A$ of $n$ elements sorted in non-decreasing order, and a search key $x$}
    \KwResult{The index of $x$ in $A$, or $-1$ if $x$ is not found}
    $low \gets 0$\;
    $high \gets n - 1$\;
    \While{$low \leq high$}{
        $mid \gets \left\lfloor\frac{low + high}{2}\right\rfloor$ \Comment*[r]{Compute the midpoint}
        \If{$A[mid] = x$}{
            \Return $mid$ \Comment*[r]{Found $x$ at index $mid$}
        }
        \If{$A[mid] < x$}{
            $low \gets mid + 1$ \Comment*[r]{$x$ must be in the right half}
        }
        \Else{
            $high \gets mid - 1$ \Comment*[r]{$x$ must be in the left half}
        }
    }
    \Return $-1$ \Comment*[r]{$x$ is not in the array}
\end{algorithm}



% Merge Sort Algorithm

\begin{algorithm}[hbt!]
    \caption{Merge Sort Algorithm}\label{alg:merge_sort}
    \KwData{An array $A$ of $n$ elements}
    \KwResult{The array $A$ sorted in non-decreasing order}
    
    \If{$n > 1$}{
        $mid \gets \left\lfloor\frac{n}{2}\right\rfloor$ \Comment*[r]{Find the middle index}
        $L \gets$ copy of the left half of $A$ from index $0$ to $mid-1$\;
        $R \gets$ copy of the right half of $A$ from index $mid$ to $n-1$\;
    
        \BlankLine
        \tcc{Recursively sort the left and right halves}
        MergeSort($L$)\;
        MergeSort($R$)\;
    
        \BlankLine
        \tcc{Merge the sorted halves back into $A$}
        $i \gets 0$\;
        $j \gets 0$\;
        $k \gets 0$\;
    
        \While{$i < \text{length of } L$ \textbf{and} $j < \text{length of } R$}{
            \If{$L[i] \leq R[j]$}{
                $A[k] \gets L[i]$\;
                $i \gets i + 1$\;
            }
            \Else{
                $A[k] \gets R[j]$\;
                $j \gets j + 1$\;
            }
            $k \gets k + 1$\;
        }
    
        \BlankLine
        \tcc{Copy any remaining elements of $L$ and $R$ into $A$}
        \While{$i < \text{length of } L$}{
            $A[k] \gets L[i]$\;
            $i \gets i + 1$\;
            $k \gets k + 1$\;
        }
    
        \While{$j < \text{length of } R$}{
            $A[k] \gets R[j]$\;
            $j \gets j + 1$\;
            $k \gets k + 1$\;
        }
    }
\end{algorithm}
    
% Largest Number
\begin{algorithm}[hbt!]
    \caption{Find the Largest Number in an Array}\label{alg:find_largest}
    \KwData{An array $A$ of $n$ elements}
    \KwResult{The largest number in the array}
    
    $largest \gets A[0]$ \Comment*[r]{Initialize largest with the first element of the array}
    
    \For{$i \gets 1$ \KwTo $n-1$}{
        \If{$A[i] > largest$}{
            $largest \gets A[i]$ \Comment*[r]{Update largest if a larger number is found}
        }
    }
    
    \Return $largest$ \Comment*[r]{The largest number in the array}
    \end{algorithm}
    

The \verb|algorithm2e| package also provides a \verb|\listofalgorithms| command that works like \verb|\listoffigures|, but for captioned algorithms:

\listofalgorithms

\end{document}
