\documentclass[11pt]{article}
\usepackage[margin=3cm]{geometry}
\usepackage{algorithm2e}
\begin{document}

Here we demonstrate how algorithms or pseudocode can be typeset using the \verb|algorithm| environment provided by the \verb|algorithm2e| package.

{\itshape You should not load the \verb|algorithm|, \verb|algpseudocode|, \verb|algcompatible|, \verb|algorithmic| packages if you have already loaded \verb|algorithm2e|.} 

Note that the command and argument syntax provided by \verb|algorithm2e| are very different from those provided by \verb|algpseudocode|. It is important to know clearly which package that you are using, and then accordingly write the relevant commands with the correct syntax.

\begin{algorithm}
$i\gets 10$\;
\eIf{$i\geq 5$}
{
    $i\gets i-1$\;
}{
    \If{$i\leq 3$}
    {
        $i\gets i+2$\;
    }
}
\end{algorithm}

Every line in your source code \textbf{must} end with \verb|\;| otherwise your algorithm will continue on the same line of text in the output. Only lines with a macro beginning a block should not end with \verb|\;|.

The above algorithm example is uncaptioned. If you need a caption for your algorithm, use \verb|\caption{...}| inside the \verb|algorithm| environment.
You can then use \verb|\label{...}| after the \verb|\caption| so that the algorithm number can be cross-referenced, e.g.~Algorithm~\ref{alg:two} and \ref{alg:three}.

By default, the \verb|plain| algorithm style is used. But if you prefer lines around the algorithm and caption, you can add the \verb|ruled| package option when loading \verb|algorithm2e|, or write \verb|\RestyleAlgo{ruled}| in your document.

\RestyleAlgo{ruled}

%% This is needed if you want to add comments in
%% your algorithm with \Comment
\SetKwComment{Comment}{/* }{ */}

\begin{algorithm}[hbt!]
\caption{An algorithm with caption}\label{alg:two}
\KwData{$n \geq 0$}
\KwResult{$y = x^n$}
$y \gets 1$\;
$X \gets x$\;
$N \gets n$\;
\While{$N \neq 0$}{
  \eIf{$N$ is even}{
    $X \gets X \times X$\;
    $N \gets \frac{N}{2} $ \Comment*[r]{This is a comment}
  }{\If{$N$ is odd}{
      $y \gets y \times X$\;
      $N \gets N - 1$\;
    }
  }
}
\end{algorithm}

The \verb|algorithm| environment is a \emph{float}, like \verb|table| and \verb|figure|, so you can add float placement modifiers \verb|[hbt!]| after \verb|\begin{algorithm}| if necessary.

\begin{algorithm}
\caption{Another algorithm with caption}\label{alg:three}
\KwData{Write here the required data}
\KwResult{Write here the expected result}
 initialization\;
 \While{While condition}{
  instructions\;
  \eIf{condition}{
   instructions1\;
   instructions2\;
   }{
   instructions3\;
  }
 }
\end{algorithm}

The \verb|algorithm2e| package also provides a \verb|\listofalgorithms| command that works like \verb|\listoffigures|, but for captioned algorithms:

\listofalgorithms


\end{document}
